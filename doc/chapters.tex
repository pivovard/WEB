\setlength{\parskip}{1em}

\chapter{Zadání}

\textbf{Standardní zadání - webové stránky konferenčního systému}

Vaším úkolem bude vytvořit webové stránky konference.  Téma konference si můžete zvolit libovolné.\\
Uživateli systému budou autoři příspěvků (vkládají abstrakty a PDF dokumenty), recenzenti příspěvků (hodnotí příspěvky) a administrátoři (spravují uživatele, přiřazují příspěvky recenzentům a rozhodují o publikování příspěvků). Každý uživatel se bude do systému přihlašovat prostřednictvím uživatelského jména a hesla. Nepřihlášený uživatel vidí pouze publikované příspěvky.\\
Nový uživatel se bude moci zaregistrovat, čímž získá status autora.\\
Přihlášený autor vidí svoje příspěvky a stav, ve kterém se nacházejí (v recenzním řízení / přijat +hodnocení / odmítnut +hodnocení). Příspěvky může přidávat, editovat a volitelně i mazat.\\
Přihlášený recenzent vidí příspěvky, které mu byly přiděleny k recenzi, a může je hodnotit (alespoň 3 kritéria). Pokud příspěvek nebyl dosud schválen, tak své hodnocení může změnit.\\
Administrátor spravuje uživatele (určuje jejich role a může uživatele zablokovat či smazat), přiřazuje neschválené příspěvky recenzentům k ohodnocení (každý příspěvek bude recenzován minimálně třemi recenzenty) a na základě recenzí rozhoduje o přijetí nebo odmítnutí příspěvku. Přijaté příspěvky jsou automaticky publikovány ve veřejné části webu.\\
Databáze musí obsahovat alespoň 3 tabulky dostatečně naplněné daty pro předvedení funkčnosti aplikace.


%%%%%%%%%%%%%%%%%%%%%%%%%%%%%%%%%%%%%%%%%%%%%%%%%%%%%%%%%%%%%%%%%%%%%%%%%%%%%%%%%%%%%%%%%%%%%%%%%%%%

\chapter{Použité technologie}


\chapter{Adresářová struktura}

\chapter{Architektura aplikace}





%%%%%%%%%%%%%%%%%%%%%%%%%%%%%%%%%%%%%%%%%%%%%%%%%%%%%%%%%%%%%%%%%%%%%%%%%%%%%%%%%%%%%%%%%%%%%%%%%%%%



\chapter{Závěr}

